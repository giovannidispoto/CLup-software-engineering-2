\documentclass[a4paper, 10pt, oneside]{article}

\usepackage[square, numbers, comma, sort&compress]{natbib}  % Use the "Natbib" style for the references in the Bibliography
\usepackage{verbatim} 
\usepackage[english]{babel}
\usepackage[utf8x]{inputenc}
\usepackage{graphicx}
\usepackage{multirow}
\usepackage{amsmath}
\usepackage{amssymb}
%\usepackage{cite}
\usepackage{booktabs}
\usepackage{epstopdf}
\usepackage{helvet} 
\renewcommand{\familydefault}{\sfdefault}
\usepackage{setspace}
\singlespacing % interlinea singola
\linespread{0.97}
\usepackage{color}
\usepackage[margin=2.5cm]{geometry}
\setlength{\parindent}{0pt}
% pacchetti aggiunti
\usepackage{comment}
\usepackage[export]{adjustbox}
\usepackage{subcaption}
\usepackage{algorithm}
\usepackage{algorithmic}
\usepackage{amsfonts}


\usepackage{titling}
\renewcommand\maketitlehooka{\null\mbox{}\vfill}
\renewcommand\maketitlehookd{\vfill\null}

\usepackage{enumerate}
\usepackage{enumitem}
\usepackage[dvipsnames]{xcolor}
\newcommand*{\lorenzo}[1]{\textcolor{BurntOrange}{#1}}
\newcommand{\yasmin}[1]{\textcolor{Red}{#1}}
\newcommand{\giovanni}[1]{\textcolor{Blue}{#1}}


\title{RASD}
\author{Yasmin Awad, Lorenzo Carpaneto, Giovanni Dispoto}
\date{October 2020}

\begin{document}

\begin{titlepage}
\vspace*{\fill}
\begin{figure}[h!]
    \centering
    \includegraphics[scale=0.5]{img/logopoli.png}
\end{figure}
\vspace{0.7em}
\begin{center}
    \Large \textbf{REQUIREMENTS ANALYSIS AND SPECIFICATION DOCUMENT}
\end{center}
\begin{center}
    \large Version 0.1
\end{center}
\begin{center}
    \large Version Date
\end{center}
\vspace{0.4em}
\begin{center}
    \Large \textbf{CLup – Customers Line-up } 
\end{center}
\vspace{-0.6em}
\begin{center}
    \normalsize Yasmin Awad, Lorenzo Carpaneto, Giovanni Dispoto
\end{center}
\vspace*{\fill}
\end{titlepage}

\normalsize


\newpage
\tableofcontents
\newpage

\section{Introduction}
\label{introduction}
%During the first part of Covid-19 pandemic was figured out the problem of manage people at grocery store. Outside these buildings roads were tons of people enquequed waiting for their turn to buy food, waiting time that could be spent in other activities.\textbf{CLup} allows to manage customers in order to minimize the time spent enqueued and also avoiding crowd inside the building, suggesting to people the right route in order to buy everything that they need, minimizing the flow inside ward departement. With CLup user can get a ticket and know at what time he should be at grocery store. The system is able to estimate time needed to user to get to grocery store and decide if he has to denqueued because doesn't have enough time for reach the position. Users can also book a slot of time to do their shopping. Using this option he need to provide to the system categories of items that he need or better the shopping list. This information is useful due to trace a correct path inside the building and also to load balancing areas. 
%\yasmin{ho provato a riscriverlo usando delle tue parti ma cambiando qualcosa}
The aim of the following documentation is to provide an overview of the project \textit{CLup - Customes Line-up}: an application whose goal is to allow customers to shop safely. During the Covid-19 pandemic a new problem arose in people's lives. In a world where social distance is no longer a choice but a necessity, it must be found a way to ensure the safety of people in their daily lives, even in places open to the public, such as grocery stores. Having to comply with health and hygiene regulations, grocery stores were not allowed to enter more than a certain number of people, with the result of the formation of very long queues outside the buildings. \textbf{CLup} allows to manage customers in order to minimize their time spent enqueued and avoiding crowd inside the building. With CLup the customer can line from home getting a ticket for the grocery store or book a specific time-slot for their needs. This documentation will illustrate a description of the system in terms of its functional and non-functional requirements. We are going to show goals, constraints, limits and principal use cases of the software. In practice, this document will provide a baseline for the project planning and estimation. The document is addressed to the developers, that will implement the requirements, and to the stakeholders, that will supervise the development process.

\subsection{Purpose}
\label{intro:purpose}
Customer Line-up (CL-up) is an application that aims to provide users with the tools to obtain access to Stores in a safe way, i.e. in order to avoid as much as possible the presence of external people (queues to enter) and to allow a number of internal people (number of people entered in the Store) lower than the allowed limit of each Store. All of this is made in order to comply as much as possible with any social distancing rules required by the competent authorities. To pursue this goal, the application provides a service both to the owners of the Stores (or whoever takes their place), that is the Managers, and to the buyers, that is the Customers. The application allows Users to create a different type of profile based on their identity, intended precisely as a Manager or Customer. Tools are provided to allow Customers to book for entry to the Stores remotely, that is for example from their own home, allowing a greater and more efficient social distancing.\\
\\
There are 2 ways offered by CLup to enter the Stores. The first concerns the collection of a Ticket that will put the Customer in a virtual queue to enter the Store on the same day. The application has also the role of alert the Customer in order to let him/her arrive in time for his/her turn, checking and keeping track of the state of the virtual queue. The second allows the Customer to book a Visit, that is an entrance to the Store on a specific day, for a specific shopping time-range. To book a Visit the Customer must also specify the category of groceries he/her is intended to buy, in order to assign him/her a map of the Store and an optimal route for his/her Visit. Furthermore, the application to be as efficient and simple as possible for users, will not only advise them a time of stay in the Store based on the user's data (that on his old time stays), but will also recommend which Stores are available or closest, always based on the user's habits.\\
\\
Regarding Managers, the application gives them the possibility to create a specialized profile in which they can request 'the creation of a Store', i.e. a virtual space that will be identified as their own Store, and through which Customers will be able to book Tickets or Visits to enter the shop itself. The Managers have in fact the task of inserting and helping the application in the creation and definition of the Store and it's rules. All this information may vary over time and must be kept up to date in order to guarantee a better service of the application.\\
\\
To guarantee entry only for Customers authorized to enter at that moment, whether they have requested a Ticket or a Visit, a QR code is checked. In practice, each Customer is associated with a QR code that will appear in his profile. At the entrance, this QR code must be scanned to verify that the Customer has actually booked the Visit for that moment or that it is his turn to enter via Ticket. Moreover, to allow the virtual queue to slide adequately, each QR code must be scanned also at the exit.\\
\\
Finally, to allow entry to those who do not have the adequate technology to use the application, a secondary option must be put in place, i.e. a fallback option. To allow anyone to collect a Ticket, we have opted for adding Totems at the entrance that have the ability to print Tickets.
\newline
\newline
\newline
{\large \textbf{1.1.1 Goals}}

\begin{enumerate}[label={G.\arabic{*}}]
    % \item \label{goal:account} Allow Users to create an account inserting the required personal data. \lorenzo{cancella}
    % \begin{enumerate}[label={\ref{goal:account}.\arabic{*}}]
    %    \item \label{goal:account:types} The User can register either as a Manager or as a Customer.
    % \end{enumerate}
    \item \label{goal:influx} Allow Managers to regulate the influx of people in their Grocery Stores, to avoid the formation of queues.
    \item \label{goal:effectiveQueue} Provide an effective queueing mechanism to enter the Store.
    % \begin{enumerate}[label={\ref{goal:influx}.\arabic{*}}]
    %    \item \label{goal:influx:qr} Monitor entrances generating a unique identifier (i.e. QR codes) for each Customer.
    % \end{enumerate}
    \item \label{goal:enqueue} Allow Customers to virtually line up for Grocery Stores requesting a Ticket from different locations and devices.
    \begin{enumerate}[label={\ref{goal:enqueue}.\arabic{*}}]
        \item \label{goal:enque:time} Provide Customers with a reasonably precise estimation of the waiting time.
        \item \label{goal:enque:alert} Alert Customers \lorenzo{dire ai clienti di partire a una determinata ora, per fare in modo che riescano ad arrivare in tempo} % to take into account the time they need to get to the shop from the place they currently are. 
        \item \label{goal:enque:fallback} Guarantee a paper ticketing service as a fallback option for people who do not have access to the required technology.
    \end{enumerate}
    \item \label{goal:visit} Allow Customers to book precise time-slots for a Visit to the Store from different locations and devices.
    \begin{enumerate}[label={\ref{goal:visit}.\arabic{*}}]
    %    \item \label{goal:visit:categories} categories e time secondo me già integrati
        \item \label{goal:visit:path} Suggest to the Customer an optimal shopping path in order to buy everything they need and minimize the flow of people in the Grocery Store.
        % \item Store and analyse information about the Customer's habits in order to be able to recommend time-slots duration and availability for specific Stores, taking into account their preferences.
        \item \label{goal:visit:infoForSuggestion} Store and analyse information about the Customer's habits in order to be able to recommend the duration of the Visit and suitable Stores for the Visit.
        \item \label{goal:visit:personalNotifications} Notify customers of available slots in a day/time range inferring information from their previous Visits.
    \item \lorenzo{Penso che dovremmo mettere anche un paio di goal per l'applicazione dal punto di vista del manager: es. deve poter scannerizzare i ticket, deve poter fornire il servizio di fallback con macchinetta. Forse accennerei anche al fatto che i manager possano aggiungere negozi all'applicazione. Senza andare troppo nel dettaglio, ma mi sembrano punti essenziali}
    \end{enumerate}
\end{enumerate}

\pagebreak

{\large \textbf{1.2 Scope}}
\lorenzo{Maybe the shared are too specific. I just do not really understand the difference that should be from scopes "shared" and requirements}

\begin{flushleft}
Following the definition originally proposed by M. Jackson and P. Zave in 1995,
we will distinguish world phenomena that are event occurring in real world from machine
that is the software to be.
World and Machine comunicate with Shared Phenomena, they could be events controlled by the world and observed by the machine or events controller by machine and observed by the world.
\end{flushleft}

\begin{itemize}
    \item {\large \textbf{World}}
    \begin{enumerate}[label={S.W.\arabic{*}}]
        \item \textbf{People going into Stores}: people going to stores to buy grocery and other products.
        \item \textbf{Queue formation}: the creation of long lines outside the stores.
        \item \textbf{People moving inside Store}: people moving into store in order to find grocery
        \yasmin{
        \item \textbf{Need to buy}: circumstance in which someone needs to enter a Store in order to buy some groceries.
        \item \textbf{Social Distancing}: circumstance in which the society imposes certain hygienic, sanitary and social distancing rules.
        }
    \end{enumerate}
    \item {\large \textbf{Shared}}
    \begin{enumerate}[label={S.S.\arabic{*}}]
        \item A Customer can register and sign up to the Service
        %\item \textbf{Registration \& Login}: a user can sign-up or login if already registered to the application.
       % \item \textbf{Choosing user type}: a user can register to the app as a manager or a customer (\lorenzo{QR scanner?}). \lorenzo{Maybe delete this points} \giovanni{quote for deleting this} \yasmin{quoto}
       % \item \textbf{QR code}: each user has a unique QR code, generated from the app as soon as they register. \lorenzo{Maybe delete this points} \giovanni{quote Lorenzo} \yasmin{forse chiamarlo QR code association ?? o generation}
        % Ticket part
        \item A logged Customer can get a ticket to virtual line
       % \item \textbf{Get ticket}: a Customer can take a ticket to form a virtual line for a specific store. \yasmin{change in ".. can get a ticket to line up for a specific Store" ??}
        \item A Customer can take a ticket at the store 
        %\item \textbf{Get paper ticket}: a Customer get a paper ticker at the store 
        \item Scan the QR code of the customer before entering into the Store
       % \item \textbf{Scan ticket}: scan the QR code of each customers before entering into the Stores. \lorenzo{Maybe delete this points} \giovanni{For me is right}
        
        
        % book part
        \item A Customer can book a visit to the store specifying categories of items
        %\item \textbf{Book a Visit}: allow Customers to book a visit to the store.
        %\item \textbf{Specify grocery categories}: specify categories of items that Customer want to buy
        \item Send notification to a Customer to remind time for the visit
        %\item \textbf{Alert customer}: remind the customer that he booked a Visit, in time to let him be in time for it. \lorenzo{Maybe delete this points} \giovanni{For me is right}
        \item Show map to the user with a built path for the visit
     %   \item \textbf{Show Map}: Guide customers through the store with a map during a Visit. \lorenzo{Maybe delete this points} \giovanni{for me is right}
        
        % manager part
        \item A Manager can upload a Store map
       % \item \textbf{Upload Map}: allow Managers to upload maps of their stores.
        \item The System built a path suggested for a Customer visit
       % \item \textbf{Build Path}: build a suggested path for the Customer, in order to minimize useless movements inside the store 
        
        \item Compute information and statistics about a Customer in order to suggest him 
        
        %\item \textbf{Build statistics}: compute the average time spent by the user into the store, and suggest it to them during the ticketing process. 
        
        %\item \textbf{Suggest time-slots}: compute suitable time-slots in which would be suitable for the customer to book a Visit to the store. 
        
        %\item \textbf{Suggest other Stores}: suggest other stores to user, with available time-slot preferred by Customer
        
        \item
    \end{enumerate}
    \item {\large \textbf{Machine}}
    \begin{enumerate}[label={S.M.\arabic{*}}]
        \item \textbf{Role differentiation}:  to grant different services to different type of user.
        \item \textbf{Time calculation}: compute the time at which each customer should be in the store when taking the ticket.
        \item \textbf{DBMS Queries}: Queries to the DBMS to retrieve data.
        
        
    \end{enumerate}
\end{itemize}


{\large \textbf{1.3 Definitions, Acronyms, Abbreviations}}
\newline \newline
\textbf{1.3.1 Definitions}

\begin{enumerate}[label={D.\arabic{*}}]
\item \label{def:store} Store: synonym of Grocery Store.
\item \label{def:manager} Manager: manager of a \textit{Store}. % TODO explain what role has in the app?
\item \label{def:customer} Customer: client of a \textit{Store}. % TODO explain what role has in the app?
\item \label{def:visit} Visit: ... % TODO
\item \label{def:securepass} Secure Password: \lorenzo{passwordformat} \giovanni{magari cercare nome policy da utilizzare.}
\item \label{def:categoriesOfGroceries} Categories of Grocery:
\item \label{def:TypeOfUser}Type of Users: the type of users can be either Manager or Customer.
\item \label{def:department}Department: macro area of the store. It can contain multiple categories of Grocery.
\item \label{def:owner}Owner: a manager who owns a specific store.
\end{enumerate}

\textbf{1.3.2 Acronyms}
\begin{enumerate}[label={A.\arabic{*}}]
\item \label{def:API} API: Application Programming Interface
\end{enumerate}
\textbf{1.3.3 Abbreviations}
\begin{enumerate}[label={AB.\arabic{*}}]
\item \label{def:UML} UML: Uniform Modeling language
\end{enumerate}

{\large \textbf{1.4 Reference Documents}}
\begin{itemize}
    \item Specification document: R\&DD Assignment AY 2020-2021
    \item IEEE 830-1993: IEEE Recommended Practice for Software Requirements Specifications
    \item Alloy documentation /link/to/doc
    \item UML documentation /link/to/doc
\end{itemize}
{\large \textbf{1.5 Revision History}}
\begin{center}
 \begin{tabular}{||c c c||}
 \hline
 Date & Version & Comments \\ [0.5ex] 
 \hline\hline
 date & 1.0 & first release \\ 
 \hline
 \hline
\end{tabular}
\end{center}
{\large \textbf{1.6 Document structure}} \newline
\newline
According to IEEE standard, the RASD is structured into 5 sections
\begin{itemize}
    \item \textbf{Introduction} contains informal presentation of the project, introducing the main goals of the S2B. Also contains the Scope with main shared phenomena.
    \item \textbf{Overall Description} contains a description of the System, including constrains and assumptions
    \item \textbf{Specific Requirements} \giovanni{brief description of the section}
    \item \textbf{Formal Analysis using Alloy} contains a formal modelling of the project, in order to proof consistency of the core functionality 
    \item \textbf{Effort Spent} show the effort spent in developing the RASD
\end{itemize}

\section{Overall Description}
\label{overallDescription}

\subsection{Product perspective}
In this section we are analyzing all the shared phenomena listed before
\begin{enumerate}
\item \textbf{Customer register to Service}: If a Customer wants to register to the service he have to provide email and password.
When the process is completed, the System generate an unique QR code associated to him, used when he goes to stores.

\item \textbf{Manager register to Service}: If a Manager wants to register has to provide his/her email and password. 

\item \textbf{Customer get a ticket}: When the customer wants to go to a Store, he needs to login in into the application, search for a Store near him and enqueuing. In this instant the System provide a time 

\item \textbf{Scan QR Code}: Before entering and exit employees scan Customers QR code in order to validate the ticket and collect information used for profiling.

\item \textbf{Customer books a visit}: When a customer books a visit he have to specify where he wants to go and choosing a time-slot available and insert the categories of items that he have to buy. 

\item \textbf{Alert Customer}: The System alert the Customer in order to remind the him that he booked a Visit or for the ticket, in time to let him be in time for it.
For doing this he have to specify what type of means of transport he use.

\item \textbf{Showing Map during the Shopping}: In order to minimize the time spent inside the Store
and for avoid too much interaction with other user, the application suggest a path to the customer using information of categories of item he wants to buy.

\item \textbf{Computing statistics of the Customer}: In order to improve the user experience, the application compute statistics of Stores visited, time spent and in general their habits. These information are used for suggest stores, time-slots and the shopping time. %\giovanni{Se invece il cliente non inserisce il tempo di visita per il ticket, c'è bisogno di prevderlo} \yasmin{se non lo inserisce non può ordinarlo, no?} \giovanni{Era relativo al problema che ci eravamo posti su far inserire il tempo di permanenza nel momento in cui si prende un ticket}
\end{enumerate}

\subsection{Product functions}
Detailed description of the product functions

\begin{enumerate}
    \item \textbf{Get a Ticket}
    \begin{flushleft}
        A Customer correctly logged in, can see all the near store. After selecting one, he/she can get a ticket. In order to get a ticket he/she have to specify the duration of the shopping and the means of transport that he/she want to use to get to store. For long term Customer the time of the shopping is inferred and suggested using historical information.
        After providing this information, the System checks that there are no ovelapped reservation.
        Before confirm, the Customer can see how much time he/she have to wait before entering in. During this phase, the System check that the user can satisfy time constrain. If not, reject the request.
        After enquequing, the System continue to keep track of the user, in order to alert him/her to depart from home.
    \end{flushleft} 
        \begin{minipage}{\linewidth}
            \centering
           \includegraphics[height=0.3\textheight, scale=0.2, keepaspectratio]{img/Get_Ticket_diagram.png}
            \captionof{figure}{Getting ticket phase}
        \end{minipage}
    
    \item \textbf{Book a Visit}
    \begin{flushleft}
       A Customer correctly logged in, can see all the near store. After selecting one, he/she can book a visit. After selecting a store he/she have to specify the duration of the visit, select a time slot available and specify the categories of grocery that he/she want to buy in order to allow the System to calculate the best path in order to maximize social distancing. The time spent inside the grocery store is inferred and suggested using historical information.
       During the visit, the Customer can open the app and see the path suggested from the System.
    \end{flushleft}
    \begin{minipage}{\linewidth}
            \centering
          \includegraphics[height=0.3\textheight, scale=0.2, keepaspectratio]{img/Booking_diagram.png}
            \captionof{figure}{Booking a visit}
        \end{minipage}
        
    \item \textbf{Add a Store}
    \begin{flushleft}
        A Manager correctly logged in can add a new  Store.
        In order to create a Store a Manager should specify the name of the store, the address, the maximum number of people that can enter inside the store according to health directives, maximum number of people for each department of the store and opening hours. During this phase the Manager could specify other Managers of the store. After creation, he/she have to specify also credentials used by the System to verify that the store exists. At the end of this phase, a Manager upload a map of the stores, reporting information of the store such as walls, departments and category of grocery for each department.
    \end{flushleft}
    \item \textbf{Manage a Store}
    \begin{flushleft}
        A Manager correctly logged in can edit information of all his stores. He/she can edit the maximum number of people according to new health directives, modify the maximum number of people inside each department, map of the stores. He/she can also modify the managers, transfer ownership or delete a store from the System.
    \end{flushleft}
    \item \textbf{Perform a visit with ticket on App}
    \begin{flushleft}
        When is the turn of a Customer enqueque, before entering, he/she have to show his/her qr code to scanner machine in order to register the visit. \giovanni{Cosa accade durante la spesa se il tempo sfora quello richiesto?}.
        Once the Customer finish his shopping, during the payment, he/she scan one more time the qr code, in order to register the exit
    \end{flushleft}
    \item \textbf{Perform a visit with paper ticket}
     \begin{flushleft}
        When the Customer arrives to the grocery stores, get a ticket from the ticket machine. On the ticket is printed the entering time. Before entering the Customer have to show the qr code to scanner machine. Once the Customer finish his shopping, during the payment, he/she scan one more time the qr code, in order to register the exit
    \end{flushleft}
    \item \textbf{Perform a booked visit}
    \begin{flushleft}
        Before entering, the Customer scan his/her qr code. When he/she is inside, he can open the app and see the calculated path to follow in order to buy everything that he/she needs. Once the Customer finish his shopping, during the payment, he/she scan one more time the qr code, in order to register the exit
    \end{flushleft}
\end{enumerate}

\subsection{User Characteristics}

\section{Specific Requirements}
\label{specificRequirements}

\begin{enumerate}[align=left]
    \item[\textbf{LOGIN \& REGISTER SERVICE}]
    \item The User must be able to register or to log in if already registered.
    \item \label{req:credentialsCheck} Check if the User credentials are valid:
    \begin{enumerate}[label={-}]
        \item \giovanni{Allow the User to insert Name and Surname} \lorenzo{TO DELETE} \yasmin{TO DELETE} %\label{req:credentialsCheck:insert} 
        \item \label{req:credentialsCheck:email} Check that the email is in the right format.
        \item \label{req:credentialsCheck:uniqueness} Check that the email that the user is providing has not been used for previous registrations.
        \item \label{req:credentialsCheck:password}Check that the password is a Secure Password.
    \end{enumerate}
    \item \label{req:confirmRegistration}If the credentials are valid, the System sends a confirmation email, to let the user activate their account. \lorenzo{nome specifico per questa procedura}
    \item \label{req:login} Allow to log in using personal credentials:
    \begin{enumerate}[label={-}]
        \item \label{req:login:checkCred}Check if credentials are valid.
        \item \label{req:login:validCred} In case the email and password inserted are correct, allow the user to access all the functionalities available for the Type of User they are.
        \item \item \label{req:login:invalidCred} If the email and/or password inserted are wrong, the system must deny the access.
        \item If the username and/or password inserted are wrong, the system must notify the user. \lorenzo{Secondo me questo é da cancellare}
    \end{enumerate}
    \item \label{req:changeMail} Allow to change email only if the new email is in a correct format.
    \item Allow to change password if the new password is a Secure Password and it is different from the previous one.
    \item \label{req:changePass}Allow to change password if it has been forgotten, through the personal email.
    \item \label{req:provideQR} Provide to each Customer a unique QR code \lorenzo{right after the registration process}.
    \begin{enumerate}[label={-}]
        \item \label{req:provideQR:uniqueID} \lorenzo{The QR code will be the identifier of the user within the System. It enables the Customer to get tickets and book Visits.} \yasmin{just the first phrase (till It enables...)}
    \end{enumerate}
    
    \item[\textbf{TICKETING SERVICE}]
    \item \label{req:requestTicket} The Customer must be allowed to request a Ticket, specifying:
     \begin{enumerate}[label={-}]
        \item \label{req:requestTicket:store} The Store where he/she wants to shop.
        \item \label{req:requestTicket:duration}The duration of his/her shopping.
        \item \label{req:requestTicket:transport}The means of transport to get to the Store.
    \end{enumerate}
    \item \label{req:systemTicket}For Customers who are requesting a Ticket, the System should:
    \begin{enumerate}[label={-}]
        \item \label{req:systemTicket:verifyDist} Verify that their time distance from the Store is such that they can arrive in time for their turn.
        \item \label{req:systemTicket:addInQueue}Insert the Customer Ticket request in the queue in order to let the Customer arrive in time for his/her shopping.
        \item \label{req:systemTicket:waitingTime} Provide to the Customer the estimated waiting time before their turn.
        \item \label{req:systemTicket:notifForDepart} Notify Customers when they should depart from home in order to arrive in time for their turn.
    \end{enumerate}
    
    \item[\textbf{BOOKING SERVICE}]
    \item \label{req:requestVisit}The Customer must be allowed to request a Visit, specifying:
    \begin{enumerate}[label={-}]
        \item \label{req:requestVisit:location}The Store where he/she wants to take a Visit.
        \item \label{req:requestVisit:date}The day of the Visit.
        \item \label{req:requestVisit:duration}The duration of his/her Visit.
        \item \label{req:requestVisit:timeslot}The time slot for the Visit.
        \item \label{req:requestVisit:categoriesToBuy}The Categories of Grocery he/her is going to buy.
    \end{enumerate}
    \item \label{req:statisticForDuration}For long time Customers, while booking a Visit, the System should infer, analyzing previous Visits of that Customer, the expected duration of the Visit.
    \item \label{req:sysProvideMap}The System should provide to the Customer a map of the Store. \lorenzo{The map should be provided by the manager [reference to successive requirement]}
    \begin{enumerate}[label={-}]
        \item \label{req:sysProvideMap:suggestPath}For Customers who have booked a Visit, the system should suggest to the Customer an optimal shopping path in order to buy everything they need and to minimize possible contacts between people inside the store.
    \end{enumerate}
    \item \label{req:seeRequests}Allow Customers to consult their pending requests.
    \item \label{req:deleteTickOrVis}The Customer must be allowed to delete a Visit or a Ticket from his/her pending requests.
    \item \label{req:thirdPartyGPS}Use third party services to enable the localization of Customers and Stores. 
    \item \lorenzo{Notify Customers when they should depart from home in order to arrive in time for their Visit.} \yasmin{TO DELETE} \giovanni{LASCIALO} \lorenzo{dont know} %\label{req:notifyCustomerToDepart}
    \item \label{req:notifyCustomerOpenSlots}Notify customers of available slots in a day-time range inferring information from their previous Visits.
    
    \item Every notification will be temporarily stored in the DBMS and it will be accessible by the Customer for a limited amount of time. \lorenzo{CHECK THIS} \yasmin{IMPRECISE}%\label{req:notificationDB}
    
    \item[\textbf{STORE MANAGEMENT SERVICE}]
    \item \label{req:man:createStore}Allow Managers to create Stores, becoming the Owner of the Store.
    \begin{enumerate}[label={-}]
        \item \label{req:man:createStore:nameAndLoc}Allow to specify name of the Store and his address
         \item \label{req:man:createStore:maxPeople}Allow Managers to specify maximum number of people who can enter the Store.
        \item \label{req:man:createStore:maxPeoplePerDep}Allow Managers to specify maximum number of people for each Department of the Store.
        \item \label{req:man:createStore:openingHours}Allow Managers to specify Store opening hours.
        \item \label{req:man:createStore:daysToBook}Allow managers to specify how many days in advance a Visit can be booked
    \end{enumerate}
    \item \label{req:man:deleteStore}Allow Owner to delete their Stores.
    \item \label{req:man:verifyOwnerCred}Verify the Store creation, acquiring specific credentials that can be used to verify the validity of the Store and of the Manager. \lorenzo{(During the creation of a Store ask for an associated PEC)}
    \item \label{req:man:updateStoreInfo}Allow the Owner and Managers to edit the store:
    \begin{enumerate}[label={-}]
         \item \label{req:man:updateStoreInfo:maxPeople}Allow Managers to edit maximum number of people who can enter the Store.
        \item \label{req:man:updateStoreInfo:maxPeoplePerDep}Allow Managers to edit maximum number of people for each Department of the Store.
        \item \label{req:man:updateStoreInfo:openingHours}Allow Managers to edit Store opening hours.
         \item \label{req:man:updateStoreInfo:addMan}Allow Owner to add other Managers to their Store(s).
        \item \label{req:man:updateStoreInfo:removeMan}Allow Owner to remove Managers from their Store(s).
        \item \label{req:man:updateStoreInfo:handOwnership}Allow Owner to hand over the ownership of the Store to one of its Managers.
        \item \label{req:man:updateStoreInfo:daysToBook}Allow managers to edit how many days in advance a Visit can be booked.
    \end{enumerate}
    
        \item \label{req:man:whenToUpdate}All the information about the Store can be updated by their Managers at anytime
    \begin{enumerate}[label={-}]
        \item \label{req:man:whenToUpdate:whenEffective}If the updates change opening hours or the maximum number of people in the store or how many days in advance a Visit can be booked, the Manager will have two options: make the updates effective within 24 hours, losing booked Visits which are not valid anymore, or make the updates effective within the current amount of how many days in advance a Visit can be booked without taking a chance on losing booked Visits.
    \end{enumerate}
    
    \item \label{req:man:uploadMap}Allow Managers to upload a map of their Stores:
    \begin{enumerate}[label={-}]
        \item \label{req:man:uploadMap:GUI}Provide a GUI to specify the structure of the Store to the System (walls, Departments, category of grocery for each Department)
        \item \label{req:man:uploadMap:specCategories}Allow the Managers to specify Categories of groceries for each Department.
        \item \label{req:man:uploadMap:specInCategories}Allow Managers to specify groceries and brands for each Department. \lorenzo{!!! let's talk about the difference between department and categories of grocery.}
    \end{enumerate}
    
    \item \label{req:scannerApp}Provide a specific \textbf{Scanner App} to scan the QR codes of the Customers and to interact with the queuing mechanism.
    \begin{enumerate}[label={-}]
        \item \label{req:scannerApp:createCode}The managers of a Store can create a unique identifier linked to the Store to input it into the Scanner app to enable its functions.
    \end{enumerate}
    
    \item \label{req:paperTicketApp}Provide a specific \textbf{Paper Tickets App} to generate paper tickets in the store and to interact with the queuing mechanism.
    \begin{enumerate}[label={-}]
        \item \label{req:paperTicketApp:func}The Paper Ticket app will allow every person to get a Paper Ticket directly from the Store.
        \item \label{req:paperTicketApp:funcPaperTick}In the Paper Tickets there will be specified the same information provided with the online Ticket and it will work the same way. The associated QR code will be automatically generated by the Paper Ticket app, and it will be different from the QR codes of each Customer (i.e. User registered in the main app).
        \item \label{req:paperTicketApp:createCode}The Managers of a Store can create a unique identifier linked to the store to input it to the Paper Ticket app to enable its functions.
    \end{enumerate}
    
    
    \item \label{req:codeUnique}Each Store has one and only one queue \yasmin{LOOK AT IT AGAIN WHILE WRITING QUEUE}
    \begin{enumerate}[label={-}]
        \item \label{req:codeUnique:visitsAndQueue}Customers who have booked the Visit will be inserted in the same queue of the ones who have got a (paper or online) Ticket, but they will have higher priority.
        \item \label{req:codeUnique:ticketInfo}Every person on a Ticket can view the time in which they should enter the Store and the duration of their shopping.
        \item Just Customers with a Ticket or a Visit will have the entering time updated in real time.
        \item \label{req:codeUnique:invalidateLate}If the Consumer will not show up in time \lorenzo{[specify what in time means]}, their ticket/visit will be considered invalid and it will be cancelled.
        \item \label{req:codeUnique:exeedDuration}A Consumer which does not go out of the shop within the duration declared will activate the Queue Scaling process and the duration of his/her visit will be extended.  \lorenzo{AGGIUNGERE DI QUANTO}
        
        \item \lorenzo{DA METTERE NELLE DEFINIZIONI O SPECIFICHE}Queue Scaling: the system checks if the number of people is less than or equal of the capacity of the store in each period of time (this will be chosen during the implementation process). If not, the system will change the entrance time of some customers to validate the aforementioned constraint. The order of the people with online tickets and booked visits in the queue will be preserved . The time of entrance can be only delayed. The time of entrance of people with paper ticket is frozen (cannot be changed).
    \end{enumerate}
    
    \item[\textbf{JUST THE MACHINE}] \lorenzo{Controlliamo se questi vadano effettivamente messi.}
    \item \label{req:periodicallyComputeSuggestion}Suggestions for Customers have to be periodically computed - at least once per day.
    \item Provide protection of data using encryption. \lorenzo{TO DELETE} \giovanni {Penso sia un po' esagerato, considerando che non abbiamo informazioni particolari. Direi che basta salvare le password salate utilizzando una buona funzione di hashing}  %label{req:encryption}
    \item Suggestions are computed by means of an artificial intelligence system. \lorenzo{TO DELETE}  \giovanni{Ho scoperto recentemente l'esistenza dei Recommender Systems che si occupano proprio di fornire suggerimenti, magari userei questo al posto di AI systems}  %label{req:MLforSuggestion}
    \item \label{req:inferringSuggestions}Suggestions are based on the habits of the Customers, inferred by the System. In a first period will be suggested Stores which are available and geographically close to the Customer (within 1 km). 
    
    
    \item \lorenzo{Segnalazioni?} \lorenzo{Abbiamo deciso che i clienti non fanno segnalazioni: TO DELETE}
\end{enumerate}


\section{Formal Analysis Using Alloy}
\label{analysisAndAlloy}
sample formal analysis using alloy

\section{Effort Spent}
\label{effort}
sample effort spent

\section{REFERENCES}
\label{references}
\renewcommand{\refname}{}
\vspace{-2.6em}
\nocite{*}
\bibliography{biblio} 
\bibliographystyle{unsrtnat}

\end{document}
